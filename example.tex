\documentclass{article}

\usepackage{tikz}

% relative positioning
\usetikzlibrary{positioning}

% braces
\usetikzlibrary{decorations.pathreplacing}

% autoencoder
\usetikzlibrary{positioning,arrows.meta,calc}

% actvation functions tikz
\usepackage{subfigure}
\usepackage{pgfplots}
\usepackage[top=3cm,left=3cm,right=3cm,bottom=3cm]{geometry}
% Scriptsize axis style.
\pgfplotsset{every axis/.append style={tick label style={/pgf/number format/fixed},font=\scriptsize,ylabel near ticks,xlabel near ticks,grid=major}}

\def\layersep{3em}
\def\transferx{9em}
\def\hiddenx{7em}
\def\hiddenxn{12em}

\begin{document}

    % perceptron
    \begin{figure}[!htp]
        \centering
        \input{perceptron.tex}
        \caption{Anatomy of a perceptron.}
        \label{fig:perceptron}
    \end{figure}

    % mlp
    \begin{figure}[!htp]
        \centering
        \input{mlp.tex}
        \caption{Anatomy of a deep feedfoward network, also called feedforward neural networks, or multilayer perceptrons (MLPs).}
        \label{fig:mlp}
    \end{figure}

    % mlp vector notation
    \begin{figure}[!htp]
        \centering
        \input{mlp-vec.tex}
        \caption{Alternative notation of a deep feedfoward network, also called feedforward neural networks, or multilayer perceptrons (MLPs).}
        \label{fig:mlp-vec}
    \end{figure}

    \input{activation.tex}
    \input{autoencoder.tex}
    \input{agent-environment.tex}
\end{document}
