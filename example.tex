\documentclass{book}

\usepackage{tikz}
\usetikzlibrary{positioning,arrows.meta,calc,decorations.pathreplacing}
%\usepackage[dvipsnames]{xcolor}

% relative positioning
%\usetikzlibrary{positioning}

% braces
%\usetikzlibrary{decorations.pathreplacing}

% autoencoder
%\usetikzlibrary{positioning,arrows.meta,calc,decorations.pathreplacing}

% actvation functions tikz
\usepackage{subfigure}
\usepackage{pgfplots}
\pgfplotsset{compat=1.16}
\usepackage[top=3cm,left=3cm,right=3cm,bottom=3cm,
%    showframe
]{geometry}
% Scriptsize axis style.
\pgfplotsset{every axis/.append style={tick label style={/pgf/number format/fixed},font=\scriptsize,ylabel near ticks,xlabel near ticks,grid=major}}

%\def\layersep{3em}
%\def\transferx{9em}
%\def\hiddenx{7em}
%\def\hiddenxn{12em}

\begin{document}

    % venn
    \begin{figure}[!htp]
        \centering
        \input{src/ai-ml-dl.tex}
        \caption{
            The relation between the field of Artificial Intelligence,
            Machine Learning, Deep Learning and Reinforcement Learning.
        }
        \label{fig:al-ml-dl}
    \end{figure}

    % perceptron
    \begin{figure}[!htp]
        \centering
        \input{src/perceptron.tex}
        \caption{Anatomy of a perceptron.}
        \label{fig:perceptron}
    \end{figure}

    % mlp
    \begin{figure}[!htp]
        \centering
        \input{src/mlp.tex}
        \caption{Anatomy of a deep feedfoward network, also called feedforward neural networks, or multilayer perceptrons (MLPs).}
        \label{fig:mlp}
    \end{figure}

    % mlp vector notation
    \begin{figure}[!htp]
        \centering
        \input{src/mlp-vec.tex}
        \caption{Alternative notation of a deep feedfoward network, also called feedforward neural networks, or multilayer perceptrons (MLPs).}
        \label{fig:mlp-vec}
    \end{figure}

    \input{src/activation.tex}
    \input{src/autoencoder.tex}
    \input{src/agent-environment.tex}
\end{document}
